% Options for packages loaded elsewhere
\PassOptionsToPackage{unicode}{hyperref}
\PassOptionsToPackage{hyphens}{url}
%
\documentclass[
]{book}
\usepackage{amsmath,amssymb}
\usepackage{lmodern}
\usepackage{iftex}
\ifPDFTeX
  \usepackage[T1]{fontenc}
  \usepackage[utf8]{inputenc}
  \usepackage{textcomp} % provide euro and other symbols
\else % if luatex or xetex
  \usepackage{unicode-math}
  \defaultfontfeatures{Scale=MatchLowercase}
  \defaultfontfeatures[\rmfamily]{Ligatures=TeX,Scale=1}
\fi
% Use upquote if available, for straight quotes in verbatim environments
\IfFileExists{upquote.sty}{\usepackage{upquote}}{}
\IfFileExists{microtype.sty}{% use microtype if available
  \usepackage[]{microtype}
  \UseMicrotypeSet[protrusion]{basicmath} % disable protrusion for tt fonts
}{}
\makeatletter
\@ifundefined{KOMAClassName}{% if non-KOMA class
  \IfFileExists{parskip.sty}{%
    \usepackage{parskip}
  }{% else
    \setlength{\parindent}{0pt}
    \setlength{\parskip}{6pt plus 2pt minus 1pt}}
}{% if KOMA class
  \KOMAoptions{parskip=half}}
\makeatother
\usepackage{xcolor}
\IfFileExists{xurl.sty}{\usepackage{xurl}}{} % add URL line breaks if available
\IfFileExists{bookmark.sty}{\usepackage{bookmark}}{\usepackage{hyperref}}
\hypersetup{
  pdftitle={A Minimal Book Example},
  pdfauthor={John Doe},
  hidelinks,
  pdfcreator={LaTeX via pandoc}}
\urlstyle{same} % disable monospaced font for URLs
\usepackage{color}
\usepackage{fancyvrb}
\newcommand{\VerbBar}{|}
\newcommand{\VERB}{\Verb[commandchars=\\\{\}]}
\DefineVerbatimEnvironment{Highlighting}{Verbatim}{commandchars=\\\{\}}
% Add ',fontsize=\small' for more characters per line
\usepackage{framed}
\definecolor{shadecolor}{RGB}{248,248,248}
\newenvironment{Shaded}{\begin{snugshade}}{\end{snugshade}}
\newcommand{\AlertTok}[1]{\textcolor[rgb]{0.94,0.16,0.16}{#1}}
\newcommand{\AnnotationTok}[1]{\textcolor[rgb]{0.56,0.35,0.01}{\textbf{\textit{#1}}}}
\newcommand{\AttributeTok}[1]{\textcolor[rgb]{0.77,0.63,0.00}{#1}}
\newcommand{\BaseNTok}[1]{\textcolor[rgb]{0.00,0.00,0.81}{#1}}
\newcommand{\BuiltInTok}[1]{#1}
\newcommand{\CharTok}[1]{\textcolor[rgb]{0.31,0.60,0.02}{#1}}
\newcommand{\CommentTok}[1]{\textcolor[rgb]{0.56,0.35,0.01}{\textit{#1}}}
\newcommand{\CommentVarTok}[1]{\textcolor[rgb]{0.56,0.35,0.01}{\textbf{\textit{#1}}}}
\newcommand{\ConstantTok}[1]{\textcolor[rgb]{0.00,0.00,0.00}{#1}}
\newcommand{\ControlFlowTok}[1]{\textcolor[rgb]{0.13,0.29,0.53}{\textbf{#1}}}
\newcommand{\DataTypeTok}[1]{\textcolor[rgb]{0.13,0.29,0.53}{#1}}
\newcommand{\DecValTok}[1]{\textcolor[rgb]{0.00,0.00,0.81}{#1}}
\newcommand{\DocumentationTok}[1]{\textcolor[rgb]{0.56,0.35,0.01}{\textbf{\textit{#1}}}}
\newcommand{\ErrorTok}[1]{\textcolor[rgb]{0.64,0.00,0.00}{\textbf{#1}}}
\newcommand{\ExtensionTok}[1]{#1}
\newcommand{\FloatTok}[1]{\textcolor[rgb]{0.00,0.00,0.81}{#1}}
\newcommand{\FunctionTok}[1]{\textcolor[rgb]{0.00,0.00,0.00}{#1}}
\newcommand{\ImportTok}[1]{#1}
\newcommand{\InformationTok}[1]{\textcolor[rgb]{0.56,0.35,0.01}{\textbf{\textit{#1}}}}
\newcommand{\KeywordTok}[1]{\textcolor[rgb]{0.13,0.29,0.53}{\textbf{#1}}}
\newcommand{\NormalTok}[1]{#1}
\newcommand{\OperatorTok}[1]{\textcolor[rgb]{0.81,0.36,0.00}{\textbf{#1}}}
\newcommand{\OtherTok}[1]{\textcolor[rgb]{0.56,0.35,0.01}{#1}}
\newcommand{\PreprocessorTok}[1]{\textcolor[rgb]{0.56,0.35,0.01}{\textit{#1}}}
\newcommand{\RegionMarkerTok}[1]{#1}
\newcommand{\SpecialCharTok}[1]{\textcolor[rgb]{0.00,0.00,0.00}{#1}}
\newcommand{\SpecialStringTok}[1]{\textcolor[rgb]{0.31,0.60,0.02}{#1}}
\newcommand{\StringTok}[1]{\textcolor[rgb]{0.31,0.60,0.02}{#1}}
\newcommand{\VariableTok}[1]{\textcolor[rgb]{0.00,0.00,0.00}{#1}}
\newcommand{\VerbatimStringTok}[1]{\textcolor[rgb]{0.31,0.60,0.02}{#1}}
\newcommand{\WarningTok}[1]{\textcolor[rgb]{0.56,0.35,0.01}{\textbf{\textit{#1}}}}
\usepackage{longtable,booktabs,array}
\usepackage{calc} % for calculating minipage widths
% Correct order of tables after \paragraph or \subparagraph
\usepackage{etoolbox}
\makeatletter
\patchcmd\longtable{\par}{\if@noskipsec\mbox{}\fi\par}{}{}
\makeatother
% Allow footnotes in longtable head/foot
\IfFileExists{footnotehyper.sty}{\usepackage{footnotehyper}}{\usepackage{footnote}}
\makesavenoteenv{longtable}
\usepackage{graphicx}
\makeatletter
\def\maxwidth{\ifdim\Gin@nat@width>\linewidth\linewidth\else\Gin@nat@width\fi}
\def\maxheight{\ifdim\Gin@nat@height>\textheight\textheight\else\Gin@nat@height\fi}
\makeatother
% Scale images if necessary, so that they will not overflow the page
% margins by default, and it is still possible to overwrite the defaults
% using explicit options in \includegraphics[width, height, ...]{}
\setkeys{Gin}{width=\maxwidth,height=\maxheight,keepaspectratio}
% Set default figure placement to htbp
\makeatletter
\def\fps@figure{htbp}
\makeatother
\setlength{\emergencystretch}{3em} % prevent overfull lines
\providecommand{\tightlist}{%
  \setlength{\itemsep}{0pt}\setlength{\parskip}{0pt}}
\setcounter{secnumdepth}{5}
\usepackage{booktabs}
\ifLuaTeX
  \usepackage{selnolig}  % disable illegal ligatures
\fi
\usepackage[]{natbib}
\bibliographystyle{apalike}

\title{A Minimal Book Example}
\author{John Doe}
\date{2022-04-25}

\usepackage{amsthm}
\newtheorem{theorem}{Theorem}[chapter]
\newtheorem{lemma}{Lemma}[chapter]
\newtheorem{corollary}{Corollary}[chapter]
\newtheorem{proposition}{Proposition}[chapter]
\newtheorem{conjecture}{Conjecture}[chapter]
\theoremstyle{definition}
\newtheorem{definition}{Definition}[chapter]
\theoremstyle{definition}
\newtheorem{example}{Example}[chapter]
\theoremstyle{definition}
\newtheorem{exercise}{Exercise}[chapter]
\theoremstyle{definition}
\newtheorem{hypothesis}{Hypothesis}[chapter]
\theoremstyle{remark}
\newtheorem*{remark}{Remark}
\newtheorem*{solution}{Solution}
\begin{document}
\maketitle

{
\setcounter{tocdepth}{1}
\tableofcontents
}
\hypertarget{about}{%
\chapter{About}\label{about}}

This is a \emph{sample} book written in \textbf{Markdown}. You can use anything that Pandoc's Markdown supports; for example, a math equation \(a^2 + b^2 = c^2\).

\hypertarget{usage}{%
\section{Usage}\label{usage}}

Each \textbf{bookdown} chapter is an .Rmd file, and each .Rmd file can contain one (and only one) chapter. A chapter \emph{must} start with a first-level heading: \texttt{\#\ A\ good\ chapter}, and can contain one (and only one) first-level heading.

Use second-level and higher headings within chapters like: \texttt{\#\#\ A\ short\ section} or \texttt{\#\#\#\ An\ even\ shorter\ section}.

The \texttt{index.Rmd} file is required, and is also your first book chapter. It will be the homepage when you render the book.

\hypertarget{render-book}{%
\section{Render book}\label{render-book}}

You can render the HTML version of this example book without changing anything:

\begin{enumerate}
\def\labelenumi{\arabic{enumi}.}
\item
  Find the \textbf{Build} pane in the RStudio IDE, and
\item
  Click on \textbf{Build Book}, then select your output format, or select ``All formats'' if you'd like to use multiple formats from the same book source files.
\end{enumerate}

Or build the book from the R console:

\begin{Shaded}
\begin{Highlighting}[]
\NormalTok{bookdown}\SpecialCharTok{::}\FunctionTok{render\_book}\NormalTok{()}
\end{Highlighting}
\end{Shaded}

To render this example to PDF as a \texttt{bookdown::pdf\_book}, you'll need to install XeLaTeX. You are recommended to install TinyTeX (which includes XeLaTeX): \url{https://yihui.org/tinytex/}.

\hypertarget{preview-book}{%
\section{Preview book}\label{preview-book}}

As you work, you may start a local server to live preview this HTML book. This preview will update as you edit the book when you save individual .Rmd files. You can start the server in a work session by using the RStudio add-in ``Preview book'', or from the R console:

\begin{Shaded}
\begin{Highlighting}[]
\NormalTok{bookdown}\SpecialCharTok{::}\FunctionTok{serve\_book}\NormalTok{()}
\end{Highlighting}
\end{Shaded}

\hypertarget{lab-1}{%
\chapter{Lab 1}\label{lab-1}}

Download the Lab1\_data.xlsx data file. This file contains fake data for a 2x3x2 repeated measures design, for 10 participants. The data is in wide format. Here is the link.

\url{https://github.com/CrumpLab/rstatsmethods/raw/master/vignettes/Stats2/Lab1_data.xlsx}

Your task is to convert the data to long format, and store the long-format data in a data.frame or tibble. Print out some of the long-form data in your lab1.Rmd, to show that you did make the appropriate conversion. For extra fun, show two different ways to solve the problem.

If you need to modify the excel by hand to help you solve the problem that is OK, just make a note of it in your lab work.

\hypertarget{solution}{%
\section{Solution}\label{solution}}

\begin{Shaded}
\begin{Highlighting}[]
\CommentTok{\# read in data}
\FunctionTok{library}\NormalTok{(readxl)}
\CommentTok{\#\textgreater{} Warning: package \textquotesingle{}readxl\textquotesingle{} was built under R version 4.1.2}

\NormalTok{wide\_data }\OtherTok{\textless{}{-}} \FunctionTok{read\_xlsx}\NormalTok{(}\StringTok{"data/Lab1\_data.xlsx"}\NormalTok{)}
\CommentTok{\#\textgreater{} New names:}
\CommentTok{\#\textgreater{} * \textasciigrave{}\textasciigrave{} {-}\textgreater{} \textasciigrave{}...1\textasciigrave{}}
\CommentTok{\#\textgreater{} * \textasciigrave{}\textasciigrave{} {-}\textgreater{} \textasciigrave{}...3\textasciigrave{}}
\CommentTok{\#\textgreater{} * \textasciigrave{}\textasciigrave{} {-}\textgreater{} \textasciigrave{}...4\textasciigrave{}}
\CommentTok{\#\textgreater{} * \textasciigrave{}\textasciigrave{} {-}\textgreater{} \textasciigrave{}...5\textasciigrave{}}
\CommentTok{\#\textgreater{} * \textasciigrave{}\textasciigrave{} {-}\textgreater{} \textasciigrave{}...6\textasciigrave{}}
\CommentTok{\#\textgreater{} * \textasciigrave{}\textasciigrave{} {-}\textgreater{} \textasciigrave{}...7\textasciigrave{}}
\CommentTok{\#\textgreater{} * \textasciigrave{}\textasciigrave{} {-}\textgreater{} \textasciigrave{}...9\textasciigrave{}}
\CommentTok{\#\textgreater{} * \textasciigrave{}\textasciigrave{} {-}\textgreater{} \textasciigrave{}...10\textasciigrave{}}
\CommentTok{\#\textgreater{} * \textasciigrave{}\textasciigrave{} {-}\textgreater{} \textasciigrave{}...11\textasciigrave{}}
\CommentTok{\#\textgreater{} * \textasciigrave{}\textasciigrave{} {-}\textgreater{} \textasciigrave{}...12\textasciigrave{}}
\CommentTok{\#\textgreater{} * \textasciigrave{}\textasciigrave{} {-}\textgreater{} \textasciigrave{}...13\textasciigrave{}}

\CommentTok{\# need to handle the column headers}

\CommentTok{\# input only column headers}
\NormalTok{wide\_headers }\OtherTok{\textless{}{-}} \FunctionTok{read\_xlsx}\NormalTok{(}\StringTok{"data/Lab1\_data.xlsx"}\NormalTok{,}
                          \AttributeTok{range =} \StringTok{"B1:M3"}\NormalTok{, }
                          \AttributeTok{col\_names=}\ConstantTok{FALSE}\NormalTok{)}
\CommentTok{\#\textgreater{} New names:}
\CommentTok{\#\textgreater{} * \textasciigrave{}\textasciigrave{} {-}\textgreater{} \textasciigrave{}...1\textasciigrave{}}
\CommentTok{\#\textgreater{} * \textasciigrave{}\textasciigrave{} {-}\textgreater{} \textasciigrave{}...2\textasciigrave{}}
\CommentTok{\#\textgreater{} * \textasciigrave{}\textasciigrave{} {-}\textgreater{} \textasciigrave{}...3\textasciigrave{}}
\CommentTok{\#\textgreater{} * \textasciigrave{}\textasciigrave{} {-}\textgreater{} \textasciigrave{}...4\textasciigrave{}}
\CommentTok{\#\textgreater{} * \textasciigrave{}\textasciigrave{} {-}\textgreater{} \textasciigrave{}...5\textasciigrave{}}
\CommentTok{\#\textgreater{} * \textasciigrave{}\textasciigrave{} {-}\textgreater{} \textasciigrave{}...6\textasciigrave{}}
\CommentTok{\#\textgreater{} * \textasciigrave{}\textasciigrave{} {-}\textgreater{} \textasciigrave{}...7\textasciigrave{}}
\CommentTok{\#\textgreater{} * \textasciigrave{}\textasciigrave{} {-}\textgreater{} \textasciigrave{}...8\textasciigrave{}}
\CommentTok{\#\textgreater{} * \textasciigrave{}\textasciigrave{} {-}\textgreater{} \textasciigrave{}...9\textasciigrave{}}
\CommentTok{\#\textgreater{} * \textasciigrave{}\textasciigrave{} {-}\textgreater{} \textasciigrave{}...10\textasciigrave{}}
\CommentTok{\#\textgreater{} * \textasciigrave{}\textasciigrave{} {-}\textgreater{} \textasciigrave{}...11\textasciigrave{}}
\CommentTok{\#\textgreater{} * \textasciigrave{}\textasciigrave{} {-}\textgreater{} \textasciigrave{}...12\textasciigrave{}}

\CommentTok{\# extract individual levels, and repeat level to fill design}
\NormalTok{IV1 }\OtherTok{\textless{}{-}} \FunctionTok{as.character}\NormalTok{(wide\_headers[}\DecValTok{1}\NormalTok{,])}
\NormalTok{IV1 }\OtherTok{\textless{}{-}}\NormalTok{ IV1[}\FunctionTok{is.na}\NormalTok{(IV1) }\SpecialCharTok{==} \ConstantTok{FALSE}\NormalTok{]}
\NormalTok{IV1 }\OtherTok{\textless{}{-}} \FunctionTok{rep}\NormalTok{(IV1, }\AttributeTok{each =} \DecValTok{6}\NormalTok{)}

\NormalTok{IV2 }\OtherTok{\textless{}{-}} \FunctionTok{as.character}\NormalTok{(wide\_headers[}\DecValTok{2}\NormalTok{,])}
\NormalTok{IV2 }\OtherTok{\textless{}{-}}\NormalTok{ IV2[}\FunctionTok{is.na}\NormalTok{(IV2) }\SpecialCharTok{==} \ConstantTok{FALSE}\NormalTok{]}
\NormalTok{IV2 }\OtherTok{\textless{}{-}} \FunctionTok{rep}\NormalTok{(IV2, }\AttributeTok{each=}\DecValTok{2}\NormalTok{)}

\NormalTok{IV3 }\OtherTok{\textless{}{-}} \FunctionTok{as.character}\NormalTok{(wide\_headers[}\DecValTok{3}\NormalTok{,])}
\NormalTok{IV3 }\OtherTok{\textless{}{-}}\NormalTok{ IV3[}\FunctionTok{is.na}\NormalTok{(IV3) }\SpecialCharTok{==} \ConstantTok{FALSE}\NormalTok{]}

\CommentTok{\# create a single row version of column headers}
\NormalTok{one\_row\_header }\OtherTok{\textless{}{-}} \FunctionTok{paste}\NormalTok{(IV1,IV2,IV3, }\AttributeTok{sep=}\StringTok{"\_"}\NormalTok{)}

\CommentTok{\# read in data again, skipping unnecessary column headers}

\NormalTok{wide\_data }\OtherTok{\textless{}{-}} \FunctionTok{read\_xlsx}\NormalTok{(}\StringTok{"data/Lab1\_data.xlsx"}\NormalTok{, }\AttributeTok{skip =} \DecValTok{2}\NormalTok{)}
\CommentTok{\#\textgreater{} New names:}
\CommentTok{\#\textgreater{} * \textasciigrave{}A\textasciigrave{} {-}\textgreater{} \textasciigrave{}A...2\textasciigrave{}}
\CommentTok{\#\textgreater{} * \textasciigrave{}B\textasciigrave{} {-}\textgreater{} \textasciigrave{}B...3\textasciigrave{}}
\CommentTok{\#\textgreater{} * \textasciigrave{}A\textasciigrave{} {-}\textgreater{} \textasciigrave{}A...4\textasciigrave{}}
\CommentTok{\#\textgreater{} * \textasciigrave{}B\textasciigrave{} {-}\textgreater{} \textasciigrave{}B...5\textasciigrave{}}
\CommentTok{\#\textgreater{} * \textasciigrave{}A\textasciigrave{} {-}\textgreater{} \textasciigrave{}A...6\textasciigrave{}}
\CommentTok{\#\textgreater{} * \textasciigrave{}B\textasciigrave{} {-}\textgreater{} \textasciigrave{}B...7\textasciigrave{}}
\CommentTok{\#\textgreater{} * \textasciigrave{}A\textasciigrave{} {-}\textgreater{} \textasciigrave{}A...8\textasciigrave{}}
\CommentTok{\#\textgreater{} * \textasciigrave{}B\textasciigrave{} {-}\textgreater{} \textasciigrave{}B...9\textasciigrave{}}
\CommentTok{\#\textgreater{} * \textasciigrave{}A\textasciigrave{} {-}\textgreater{} \textasciigrave{}A...10\textasciigrave{}}
\CommentTok{\#\textgreater{} * \textasciigrave{}B\textasciigrave{} {-}\textgreater{} \textasciigrave{}B...11\textasciigrave{}}
\CommentTok{\#\textgreater{} * \textasciigrave{}A\textasciigrave{} {-}\textgreater{} \textasciigrave{}A...12\textasciigrave{}}
\CommentTok{\#\textgreater{} * \textasciigrave{}B\textasciigrave{} {-}\textgreater{} \textasciigrave{}B...13\textasciigrave{}}

\CommentTok{\# replace names with new column headers }

\FunctionTok{names}\NormalTok{(wide\_data)[}\DecValTok{2}\SpecialCharTok{:}\DecValTok{13}\NormalTok{] }\OtherTok{\textless{}{-}}\NormalTok{ one\_row\_header}

\CommentTok{\# use pivot\_longer to convert to long}

\FunctionTok{library}\NormalTok{(tidyr)}
\CommentTok{\#\textgreater{} Warning: package \textquotesingle{}tidyr\textquotesingle{} was built under R version 4.1.2}

\NormalTok{long\_data }\OtherTok{\textless{}{-}}\NormalTok{ wide\_data }\SpecialCharTok{\%\textgreater{}\%} \FunctionTok{pivot\_longer}\NormalTok{(}
  \AttributeTok{cols =} \DecValTok{2}\SpecialCharTok{:}\DecValTok{13}\NormalTok{,}
  \AttributeTok{names\_to =} \FunctionTok{c}\NormalTok{(}\StringTok{"Loudness"}\NormalTok{,}\StringTok{"Time"}\NormalTok{,}\StringTok{"Letter"}\NormalTok{),}
  \AttributeTok{names\_pattern =} \StringTok{"(.*)\_(.*)\_(.*)"}\NormalTok{,}
  \AttributeTok{values\_to =} \StringTok{"DV"}
\NormalTok{)}

\NormalTok{knitr}\SpecialCharTok{::}\FunctionTok{kable}\NormalTok{(}\FunctionTok{head}\NormalTok{(long\_data))}
\end{Highlighting}
\end{Shaded}

\begin{tabular}{r|l|l|l|r}
\hline
Participant & Loudness & Time & Letter & DV\\
\hline
1 & Noisy & Morning & A & 61\\
\hline
1 & Noisy & Morning & B & 77\\
\hline
1 & Noisy & Afternoon & A & 97\\
\hline
1 & Noisy & Afternoon & B & 97\\
\hline
1 & Noisy & Evening & A & 89\\
\hline
1 & Noisy & Evening & B & 94\\
\hline
\end{tabular}

\hypertarget{other-solutions}{%
\section{other solutions}\label{other-solutions}}

\begin{Shaded}
\begin{Highlighting}[]
\CommentTok{\# create IVs by hand}
\NormalTok{Loudness }\OtherTok{\textless{}{-}} \FunctionTok{rep}\NormalTok{( }\FunctionTok{rep}\NormalTok{(}\FunctionTok{c}\NormalTok{(}\StringTok{"Noisy"}\NormalTok{,}\StringTok{"Quiet"}\NormalTok{),}\AttributeTok{each=}\DecValTok{6}\NormalTok{), }\DecValTok{10}\NormalTok{)}
\NormalTok{Time     }\OtherTok{\textless{}{-}} \FunctionTok{rep}\NormalTok{( }\FunctionTok{rep}\NormalTok{(}\FunctionTok{rep}\NormalTok{(}\FunctionTok{c}\NormalTok{(}\StringTok{"Morning"}\NormalTok{,}\StringTok{"Afternoon"}\NormalTok{,}\StringTok{"Evening"}\NormalTok{),}\AttributeTok{each=}\DecValTok{2}\NormalTok{),}\DecValTok{2}\NormalTok{), }\DecValTok{10}\NormalTok{)}
\NormalTok{Letter   }\OtherTok{\textless{}{-}} \FunctionTok{rep}\NormalTok{( }\FunctionTok{rep}\NormalTok{(}\FunctionTok{c}\NormalTok{(}\StringTok{"A"}\NormalTok{,}\StringTok{"B"}\NormalTok{),}\DecValTok{6}\NormalTok{), }\DecValTok{10}\NormalTok{)}
\NormalTok{Participant }\OtherTok{\textless{}{-}} \FunctionTok{rep}\NormalTok{(}\DecValTok{1}\SpecialCharTok{:}\DecValTok{10}\NormalTok{, }\AttributeTok{each =} \DecValTok{12}\NormalTok{)}

\CommentTok{\#load rectangle containing data}
\NormalTok{wide\_data }\OtherTok{\textless{}{-}} \FunctionTok{read\_xlsx}\NormalTok{(}\StringTok{"data/Lab1\_data.xlsx"}\NormalTok{,}
                          \AttributeTok{range =} \StringTok{"B4:M13"}\NormalTok{, }
                          \AttributeTok{col\_names=}\ConstantTok{FALSE}\NormalTok{)}
\CommentTok{\#\textgreater{} New names:}
\CommentTok{\#\textgreater{} * \textasciigrave{}\textasciigrave{} {-}\textgreater{} \textasciigrave{}...1\textasciigrave{}}
\CommentTok{\#\textgreater{} * \textasciigrave{}\textasciigrave{} {-}\textgreater{} \textasciigrave{}...2\textasciigrave{}}
\CommentTok{\#\textgreater{} * \textasciigrave{}\textasciigrave{} {-}\textgreater{} \textasciigrave{}...3\textasciigrave{}}
\CommentTok{\#\textgreater{} * \textasciigrave{}\textasciigrave{} {-}\textgreater{} \textasciigrave{}...4\textasciigrave{}}
\CommentTok{\#\textgreater{} * \textasciigrave{}\textasciigrave{} {-}\textgreater{} \textasciigrave{}...5\textasciigrave{}}
\CommentTok{\#\textgreater{} * \textasciigrave{}\textasciigrave{} {-}\textgreater{} \textasciigrave{}...6\textasciigrave{}}
\CommentTok{\#\textgreater{} * \textasciigrave{}\textasciigrave{} {-}\textgreater{} \textasciigrave{}...7\textasciigrave{}}
\CommentTok{\#\textgreater{} * \textasciigrave{}\textasciigrave{} {-}\textgreater{} \textasciigrave{}...8\textasciigrave{}}
\CommentTok{\#\textgreater{} * \textasciigrave{}\textasciigrave{} {-}\textgreater{} \textasciigrave{}...9\textasciigrave{}}
\CommentTok{\#\textgreater{} * \textasciigrave{}\textasciigrave{} {-}\textgreater{} \textasciigrave{}...10\textasciigrave{}}
\CommentTok{\#\textgreater{} * \textasciigrave{}\textasciigrave{} {-}\textgreater{} \textasciigrave{}...11\textasciigrave{}}
\CommentTok{\#\textgreater{} * \textasciigrave{}\textasciigrave{} {-}\textgreater{} \textasciigrave{}...12\textasciigrave{}}

\CommentTok{\# convert matrix to a single vector (concatenate)}
\NormalTok{long\_dv }\OtherTok{\textless{}{-}} \FunctionTok{c}\NormalTok{(}\FunctionTok{t}\NormalTok{(}\FunctionTok{as.matrix}\NormalTok{(wide\_data)))}

\CommentTok{\#assemble data.frame}

\NormalTok{long\_data }\OtherTok{\textless{}{-}} \FunctionTok{data.frame}\NormalTok{(Participant,}
\NormalTok{                        Loudness,}
\NormalTok{                        Time, }
\NormalTok{                        Letter,}
                        \AttributeTok{DV=}\NormalTok{long\_dv)}

\FunctionTok{head}\NormalTok{(long\_data)}
\CommentTok{\#\textgreater{}   Participant Loudness      Time Letter DV}
\CommentTok{\#\textgreater{} 1           1    Noisy   Morning      A 61}
\CommentTok{\#\textgreater{} 2           1    Noisy   Morning      B 77}
\CommentTok{\#\textgreater{} 3           1    Noisy Afternoon      A 97}
\CommentTok{\#\textgreater{} 4           1    Noisy Afternoon      B 97}
\CommentTok{\#\textgreater{} 5           1    Noisy   Evening      A 89}
\CommentTok{\#\textgreater{} 6           1    Noisy   Evening      B 94}
\end{Highlighting}
\end{Shaded}

Using loops and logic, and a minimum of other functions

\begin{Shaded}
\begin{Highlighting}[]
\CommentTok{\# load data}

\NormalTok{wide\_data }\OtherTok{\textless{}{-}} \FunctionTok{read\_xlsx}\NormalTok{(}\StringTok{"data/Lab1\_data.xlsx"}\NormalTok{,}\AttributeTok{col\_names =} \ConstantTok{FALSE}\NormalTok{)}
\CommentTok{\#\textgreater{} New names:}
\CommentTok{\#\textgreater{} * \textasciigrave{}\textasciigrave{} {-}\textgreater{} \textasciigrave{}...1\textasciigrave{}}
\CommentTok{\#\textgreater{} * \textasciigrave{}\textasciigrave{} {-}\textgreater{} \textasciigrave{}...2\textasciigrave{}}
\CommentTok{\#\textgreater{} * \textasciigrave{}\textasciigrave{} {-}\textgreater{} \textasciigrave{}...3\textasciigrave{}}
\CommentTok{\#\textgreater{} * \textasciigrave{}\textasciigrave{} {-}\textgreater{} \textasciigrave{}...4\textasciigrave{}}
\CommentTok{\#\textgreater{} * \textasciigrave{}\textasciigrave{} {-}\textgreater{} \textasciigrave{}...5\textasciigrave{}}
\CommentTok{\#\textgreater{} * \textasciigrave{}\textasciigrave{} {-}\textgreater{} \textasciigrave{}...6\textasciigrave{}}
\CommentTok{\#\textgreater{} * \textasciigrave{}\textasciigrave{} {-}\textgreater{} \textasciigrave{}...7\textasciigrave{}}
\CommentTok{\#\textgreater{} * \textasciigrave{}\textasciigrave{} {-}\textgreater{} \textasciigrave{}...8\textasciigrave{}}
\CommentTok{\#\textgreater{} * \textasciigrave{}\textasciigrave{} {-}\textgreater{} \textasciigrave{}...9\textasciigrave{}}
\CommentTok{\#\textgreater{} * \textasciigrave{}\textasciigrave{} {-}\textgreater{} \textasciigrave{}...10\textasciigrave{}}
\CommentTok{\#\textgreater{} * \textasciigrave{}\textasciigrave{} {-}\textgreater{} \textasciigrave{}...11\textasciigrave{}}
\CommentTok{\#\textgreater{} * \textasciigrave{}\textasciigrave{} {-}\textgreater{} \textasciigrave{}...12\textasciigrave{}}
\CommentTok{\#\textgreater{} * \textasciigrave{}\textasciigrave{} {-}\textgreater{} \textasciigrave{}...13\textasciigrave{}}
\NormalTok{wide\_data }\OtherTok{\textless{}{-}} \FunctionTok{as.data.frame}\NormalTok{(wide\_data)}

\CommentTok{\# Create vectors of level names for each IV}
\CommentTok{\# use a for loop to process the first three rows of wide\_data}

\NormalTok{Loudness }\OtherTok{\textless{}{-}} \FunctionTok{c}\NormalTok{()}
\NormalTok{Time }\OtherTok{\textless{}{-}} \FunctionTok{c}\NormalTok{()}
\NormalTok{Letter }\OtherTok{\textless{}{-}} \FunctionTok{c}\NormalTok{()}
\ControlFlowTok{for}\NormalTok{ (i }\ControlFlowTok{in} \DecValTok{2}\SpecialCharTok{:}\DecValTok{13}\NormalTok{) \{}
  
  \ControlFlowTok{if}\NormalTok{ ( }\FunctionTok{is.na}\NormalTok{(wide\_data[}\DecValTok{1}\NormalTok{,i]) }\SpecialCharTok{==} \ConstantTok{FALSE}\NormalTok{ ) Loudness[i}\DecValTok{{-}1}\NormalTok{] }\OtherTok{\textless{}{-}}\NormalTok{ wide\_data[}\DecValTok{1}\NormalTok{,i]}
  \ControlFlowTok{if}\NormalTok{ ( }\FunctionTok{is.na}\NormalTok{(wide\_data[}\DecValTok{1}\NormalTok{,i]) }\SpecialCharTok{==} \ConstantTok{TRUE}\NormalTok{ ) Loudness[i}\DecValTok{{-}1}\NormalTok{] }\OtherTok{\textless{}{-}}\NormalTok{ Loudness[i}\DecValTok{{-}2}\NormalTok{]}
  
  \ControlFlowTok{if}\NormalTok{ ( }\FunctionTok{is.na}\NormalTok{(wide\_data[}\DecValTok{2}\NormalTok{,i]) }\SpecialCharTok{==} \ConstantTok{FALSE}\NormalTok{ ) Time[i}\DecValTok{{-}1}\NormalTok{] }\OtherTok{\textless{}{-}}\NormalTok{ wide\_data[}\DecValTok{2}\NormalTok{,i]}
  \ControlFlowTok{if}\NormalTok{ ( }\FunctionTok{is.na}\NormalTok{(wide\_data[}\DecValTok{2}\NormalTok{,i]) }\SpecialCharTok{==} \ConstantTok{TRUE}\NormalTok{ ) Time[i}\DecValTok{{-}1}\NormalTok{] }\OtherTok{\textless{}{-}}\NormalTok{ Time[i}\DecValTok{{-}2}\NormalTok{]}
  
  \ControlFlowTok{if}\NormalTok{ ( }\FunctionTok{is.na}\NormalTok{(wide\_data[}\DecValTok{3}\NormalTok{,i]) }\SpecialCharTok{==} \ConstantTok{FALSE}\NormalTok{ ) Letter[i}\DecValTok{{-}1}\NormalTok{] }\OtherTok{\textless{}{-}}\NormalTok{ wide\_data[}\DecValTok{3}\NormalTok{,i]}
  \ControlFlowTok{if}\NormalTok{ ( }\FunctionTok{is.na}\NormalTok{(wide\_data[}\DecValTok{3}\NormalTok{,i]) }\SpecialCharTok{==} \ConstantTok{TRUE}\NormalTok{ ) Letter[i}\DecValTok{{-}1}\NormalTok{] }\OtherTok{\textless{}{-}}\NormalTok{ Letter[i}\DecValTok{{-}2}\NormalTok{]}
\NormalTok{\}}

\CommentTok{\# Create a long data frame using a for loop}

\NormalTok{long\_data }\OtherTok{\textless{}{-}}  \FunctionTok{data.frame}\NormalTok{()}

\ControlFlowTok{for}\NormalTok{(i }\ControlFlowTok{in} \DecValTok{4}\SpecialCharTok{:}\DecValTok{13}\NormalTok{)\{ }\CommentTok{\# rows}
  \ControlFlowTok{for}\NormalTok{(j }\ControlFlowTok{in} \DecValTok{2}\SpecialCharTok{:}\DecValTok{13}\NormalTok{) \{ }\CommentTok{\# columns}
\NormalTok{    temp\_row }\OtherTok{\textless{}{-}} \FunctionTok{data.frame}\NormalTok{(}\AttributeTok{Participant =}\NormalTok{ wide\_data[i,}\DecValTok{1}\NormalTok{],}
                           \AttributeTok{Loudness =}\NormalTok{ Loudness[j}\DecValTok{{-}1}\NormalTok{],}
                           \AttributeTok{Time =}\NormalTok{ Time[j}\DecValTok{{-}1}\NormalTok{],}
                           \AttributeTok{Letter =}\NormalTok{ Letter[j}\DecValTok{{-}1}\NormalTok{],}
                           \AttributeTok{DV =}\NormalTok{ wide\_data[i,j])}
\NormalTok{    long\_data }\OtherTok{\textless{}{-}} \FunctionTok{rbind}\NormalTok{(long\_data,temp\_row)}
\NormalTok{  \}}
\NormalTok{\}}

\FunctionTok{head}\NormalTok{(long\_data)}
\CommentTok{\#\textgreater{}   Participant Loudness      Time Letter DV}
\CommentTok{\#\textgreater{} 1           1    Noisy   Morning      A 61}
\CommentTok{\#\textgreater{} 2           1    Noisy   Morning      B 77}
\CommentTok{\#\textgreater{} 3           1    Noisy Afternoon      A 97}
\CommentTok{\#\textgreater{} 4           1    Noisy Afternoon      B 97}
\CommentTok{\#\textgreater{} 5           1    Noisy   Evening      A 89}
\CommentTok{\#\textgreater{} 6           1    Noisy   Evening      B 94}
\end{Highlighting}
\end{Shaded}

Using the \texttt{zoo:na.locf} function to fill level names to the right. This uses only a few lines, but it comes at the expense of low readability, and lots of nested function calls that are hard to parse and understand.

\begin{Shaded}
\begin{Highlighting}[]
\NormalTok{wide\_data }\OtherTok{\textless{}{-}} \FunctionTok{as.data.frame}\NormalTok{(}\FunctionTok{read\_xlsx}\NormalTok{(}\StringTok{"data/Lab1\_data.xlsx"}\NormalTok{,}\AttributeTok{col\_names =} \ConstantTok{FALSE}\NormalTok{))}
\CommentTok{\#\textgreater{} New names:}
\CommentTok{\#\textgreater{} * \textasciigrave{}\textasciigrave{} {-}\textgreater{} \textasciigrave{}...1\textasciigrave{}}
\CommentTok{\#\textgreater{} * \textasciigrave{}\textasciigrave{} {-}\textgreater{} \textasciigrave{}...2\textasciigrave{}}
\CommentTok{\#\textgreater{} * \textasciigrave{}\textasciigrave{} {-}\textgreater{} \textasciigrave{}...3\textasciigrave{}}
\CommentTok{\#\textgreater{} * \textasciigrave{}\textasciigrave{} {-}\textgreater{} \textasciigrave{}...4\textasciigrave{}}
\CommentTok{\#\textgreater{} * \textasciigrave{}\textasciigrave{} {-}\textgreater{} \textasciigrave{}...5\textasciigrave{}}
\CommentTok{\#\textgreater{} * \textasciigrave{}\textasciigrave{} {-}\textgreater{} \textasciigrave{}...6\textasciigrave{}}
\CommentTok{\#\textgreater{} * \textasciigrave{}\textasciigrave{} {-}\textgreater{} \textasciigrave{}...7\textasciigrave{}}
\CommentTok{\#\textgreater{} * \textasciigrave{}\textasciigrave{} {-}\textgreater{} \textasciigrave{}...8\textasciigrave{}}
\CommentTok{\#\textgreater{} * \textasciigrave{}\textasciigrave{} {-}\textgreater{} \textasciigrave{}...9\textasciigrave{}}
\CommentTok{\#\textgreater{} * \textasciigrave{}\textasciigrave{} {-}\textgreater{} \textasciigrave{}...10\textasciigrave{}}
\CommentTok{\#\textgreater{} * \textasciigrave{}\textasciigrave{} {-}\textgreater{} \textasciigrave{}...11\textasciigrave{}}
\CommentTok{\#\textgreater{} * \textasciigrave{}\textasciigrave{} {-}\textgreater{} \textasciigrave{}...12\textasciigrave{}}
\CommentTok{\#\textgreater{} * \textasciigrave{}\textasciigrave{} {-}\textgreater{} \textasciigrave{}...13\textasciigrave{}}
\NormalTok{the\_scores }\OtherTok{\textless{}{-}}\NormalTok{ wide\_data[}\DecValTok{4}\SpecialCharTok{:}\DecValTok{13}\NormalTok{,}\DecValTok{1}\SpecialCharTok{:}\DecValTok{13}\NormalTok{]}
\FunctionTok{names}\NormalTok{(the\_scores) }\OtherTok{\textless{}{-}} \FunctionTok{c}\NormalTok{(wide\_data[}\DecValTok{3}\NormalTok{,}\DecValTok{1}\NormalTok{],}\FunctionTok{apply}\NormalTok{(zoo}\SpecialCharTok{::}\FunctionTok{na.locf}\NormalTok{(}\FunctionTok{t}\NormalTok{(wide\_data[}\DecValTok{1}\SpecialCharTok{:}\DecValTok{3}\NormalTok{,}\DecValTok{2}\SpecialCharTok{:}\DecValTok{13}\NormalTok{])),}\DecValTok{1}\NormalTok{,paste, }\AttributeTok{collapse=}\StringTok{"\_"}\NormalTok{))}

\NormalTok{long\_data }\OtherTok{\textless{}{-}}\NormalTok{ the\_scores }\SpecialCharTok{\%\textgreater{}\%} \FunctionTok{pivot\_longer}\NormalTok{(}
  \AttributeTok{cols =} \DecValTok{2}\SpecialCharTok{:}\DecValTok{13}\NormalTok{,}
  \AttributeTok{names\_to =} \FunctionTok{c}\NormalTok{(}\StringTok{"Loudness"}\NormalTok{,}\StringTok{"Time"}\NormalTok{,}\StringTok{"Letter"}\NormalTok{),}
  \AttributeTok{names\_pattern =} \StringTok{"(.*)\_(.*)\_(.*)"}\NormalTok{,}
  \AttributeTok{values\_to =} \StringTok{"DV"}
\NormalTok{)}

\FunctionTok{head}\NormalTok{(long\_data)}
\CommentTok{\#\textgreater{} \# A tibble: 6 x 5}
\CommentTok{\#\textgreater{}   Participant Loudness Time      Letter DV   }
\CommentTok{\#\textgreater{}   \textless{}chr\textgreater{}       \textless{}chr\textgreater{}    \textless{}chr\textgreater{}     \textless{}chr\textgreater{}  \textless{}chr\textgreater{}}
\CommentTok{\#\textgreater{} 1 1           Noisy    Morning   A      61   }
\CommentTok{\#\textgreater{} 2 1           Noisy    Morning   B      77   }
\CommentTok{\#\textgreater{} 3 1           Noisy    Afternoon A      97   }
\CommentTok{\#\textgreater{} 4 1           Noisy    Afternoon B      97   }
\CommentTok{\#\textgreater{} 5 1           Noisy    Evening   A      89   }
\CommentTok{\#\textgreater{} 6 1           Noisy    Evening   B      94}
\end{Highlighting}
\end{Shaded}

\hypertarget{cross}{%
\chapter{Cross-references}\label{cross}}

Cross-references make it easier for your readers to find and link to elements in your book.

\hypertarget{chapters-and-sub-chapters}{%
\section{Chapters and sub-chapters}\label{chapters-and-sub-chapters}}

There are two steps to cross-reference any heading:

\begin{enumerate}
\def\labelenumi{\arabic{enumi}.}
\tightlist
\item
  Label the heading: \texttt{\#\ Hello\ world\ \{\#nice-label\}}.

  \begin{itemize}
  \tightlist
  \item
    Leave the label off if you like the automated heading generated based on your heading title: for example, \texttt{\#\ Hello\ world} = \texttt{\#\ Hello\ world\ \{\#hello-world\}}.
  \item
    To label an un-numbered heading, use: \texttt{\#\ Hello\ world\ \{-\#nice-label\}} or \texttt{\{\#\ Hello\ world\ .unnumbered\}}.
  \end{itemize}
\item
  Next, reference the labeled heading anywhere in the text using \texttt{\textbackslash{}@ref(nice-label)}; for example, please see Chapter \ref{cross}.

  \begin{itemize}
  \tightlist
  \item
    If you prefer text as the link instead of a numbered reference use: \protect\hyperlink{cross}{any text you want can go here}.
  \end{itemize}
\end{enumerate}

\hypertarget{captioned-figures-and-tables}{%
\section{Captioned figures and tables}\label{captioned-figures-and-tables}}

Figures and tables \emph{with captions} can also be cross-referenced from elsewhere in your book using \texttt{\textbackslash{}@ref(fig:chunk-label)} and \texttt{\textbackslash{}@ref(tab:chunk-label)}, respectively.

See Figure \ref{fig:nice-fig}.

\begin{Shaded}
\begin{Highlighting}[]
\FunctionTok{par}\NormalTok{(}\AttributeTok{mar =} \FunctionTok{c}\NormalTok{(}\DecValTok{4}\NormalTok{, }\DecValTok{4}\NormalTok{, .}\DecValTok{1}\NormalTok{, .}\DecValTok{1}\NormalTok{))}
\FunctionTok{plot}\NormalTok{(pressure, }\AttributeTok{type =} \StringTok{\textquotesingle{}b\textquotesingle{}}\NormalTok{, }\AttributeTok{pch =} \DecValTok{19}\NormalTok{)}
\end{Highlighting}
\end{Shaded}

\begin{figure}

{\centering \includegraphics[width=0.8\linewidth]{02-cross-refs_files/figure-latex/nice-fig-1} 

}

\caption{Here is a nice figure!}\label{fig:nice-fig}
\end{figure}

Don't miss Table \ref{tab:nice-tab}.

\begin{Shaded}
\begin{Highlighting}[]
\NormalTok{knitr}\SpecialCharTok{::}\FunctionTok{kable}\NormalTok{(}
  \FunctionTok{head}\NormalTok{(pressure, }\DecValTok{10}\NormalTok{), }\AttributeTok{caption =} \StringTok{\textquotesingle{}Here is a nice table!\textquotesingle{}}\NormalTok{,}
  \AttributeTok{booktabs =} \ConstantTok{TRUE}
\NormalTok{)}
\end{Highlighting}
\end{Shaded}

\begin{table}

\caption{\label{tab:nice-tab}Here is a nice table!}
\centering
\begin{tabular}[t]{rr}
\toprule
temperature & pressure\\
\midrule
0 & 0.0002\\
20 & 0.0012\\
40 & 0.0060\\
60 & 0.0300\\
80 & 0.0900\\
\addlinespace
100 & 0.2700\\
120 & 0.7500\\
140 & 1.8500\\
160 & 4.2000\\
180 & 8.8000\\
\bottomrule
\end{tabular}
\end{table}

\hypertarget{parts}{%
\chapter{Parts}\label{parts}}

You can add parts to organize one or more book chapters together. Parts can be inserted at the top of an .Rmd file, before the first-level chapter heading in that same file.

Add a numbered part: \texttt{\#\ (PART)\ Act\ one\ \{-\}} (followed by \texttt{\#\ A\ chapter})

Add an unnumbered part: \texttt{\#\ (PART\textbackslash{}*)\ Act\ one\ \{-\}} (followed by \texttt{\#\ A\ chapter})

Add an appendix as a special kind of un-numbered part: \texttt{\#\ (APPENDIX)\ Other\ stuff\ \{-\}} (followed by \texttt{\#\ A\ chapter}). Chapters in an appendix are prepended with letters instead of numbers.

\hypertarget{footnotes-and-citations}{%
\chapter{Footnotes and citations}\label{footnotes-and-citations}}

\hypertarget{footnotes}{%
\section{Footnotes}\label{footnotes}}

Footnotes are put inside the square brackets after a caret \texttt{\^{}{[}{]}}. Like this one \footnote{This is a footnote.}.

\hypertarget{citations}{%
\section{Citations}\label{citations}}

Reference items in your bibliography file(s) using \texttt{@key}.

For example, we are using the \textbf{bookdown} package \citep{R-bookdown} (check out the last code chunk in index.Rmd to see how this citation key was added) in this sample book, which was built on top of R Markdown and \textbf{knitr} \citep{xie2015} (this citation was added manually in an external file book.bib).
Note that the \texttt{.bib} files need to be listed in the index.Rmd with the YAML \texttt{bibliography} key.

The \texttt{bs4\_book} theme makes footnotes appear inline when you click on them. In this example book, we added \texttt{csl:\ chicago-fullnote-bibliography.csl} to the \texttt{index.Rmd} YAML, and include the \texttt{.csl} file. To download a new style, we recommend: \url{https://www.zotero.org/styles/}

The RStudio Visual Markdown Editor can also make it easier to insert citations: \url{https://rstudio.github.io/visual-markdown-editing/\#/citations}

\hypertarget{blocks}{%
\chapter{Blocks}\label{blocks}}

\hypertarget{equations}{%
\section{Equations}\label{equations}}

Here is an equation.

\begin{equation} 
  f\left(k\right) = \binom{n}{k} p^k\left(1-p\right)^{n-k}
  \label{eq:binom}
\end{equation}

You may refer to using \texttt{\textbackslash{}@ref(eq:binom)}, like see Equation \eqref{eq:binom}.

\hypertarget{theorems-and-proofs}{%
\section{Theorems and proofs}\label{theorems-and-proofs}}

Labeled theorems can be referenced in text using \texttt{\textbackslash{}@ref(thm:tri)}, for example, check out this smart theorem \ref{thm:tri}.

\begin{theorem}
\protect\hypertarget{thm:tri}{}\label{thm:tri}For a right triangle, if \(c\) denotes the \emph{length} of the hypotenuse
and \(a\) and \(b\) denote the lengths of the \textbf{other} two sides, we have
\[a^2 + b^2 = c^2\]
\end{theorem}

Read more here \url{https://bookdown.org/yihui/bookdown/markdown-extensions-by-bookdown.html}.

\hypertarget{callout-blocks}{%
\section{Callout blocks}\label{callout-blocks}}

The \texttt{bs4\_book} theme also includes special callout blocks, like this \texttt{.rmdnote}.

You can use \textbf{markdown} inside a block.

\begin{Shaded}
\begin{Highlighting}[]
\FunctionTok{head}\NormalTok{(beaver1, }\AttributeTok{n =} \DecValTok{5}\NormalTok{)}
\CommentTok{\#\textgreater{}   day time  temp activ}
\CommentTok{\#\textgreater{} 1 346  840 36.33     0}
\CommentTok{\#\textgreater{} 2 346  850 36.34     0}
\CommentTok{\#\textgreater{} 3 346  900 36.35     0}
\CommentTok{\#\textgreater{} 4 346  910 36.42     0}
\CommentTok{\#\textgreater{} 5 346  920 36.55     0}
\end{Highlighting}
\end{Shaded}

It is up to the user to define the appearance of these blocks for LaTeX output.

You may also use: \texttt{.rmdcaution}, \texttt{.rmdimportant}, \texttt{.rmdtip}, or \texttt{.rmdwarning} as the block name.

The R Markdown Cookbook provides more help on how to use custom blocks to design your own callouts: \url{https://bookdown.org/yihui/rmarkdown-cookbook/custom-blocks.html}

\hypertarget{sharing-your-book}{%
\chapter{Sharing your book}\label{sharing-your-book}}

\hypertarget{publishing}{%
\section{Publishing}\label{publishing}}

HTML books can be published online, see: \url{https://bookdown.org/yihui/bookdown/publishing.html}

\hypertarget{pages}{%
\section{404 pages}\label{pages}}

By default, users will be directed to a 404 page if they try to access a webpage that cannot be found. If you'd like to customize your 404 page instead of using the default, you may add either a \texttt{\_404.Rmd} or \texttt{\_404.md} file to your project root and use code and/or Markdown syntax.

\hypertarget{metadata-for-sharing}{%
\section{Metadata for sharing}\label{metadata-for-sharing}}

Bookdown HTML books will provide HTML metadata for social sharing on platforms like Twitter, Facebook, and LinkedIn, using information you provide in the \texttt{index.Rmd} YAML. To setup, set the \texttt{url} for your book and the path to your \texttt{cover-image} file. Your book's \texttt{title} and \texttt{description} are also used.

This \texttt{bs4\_book} provides enhanced metadata for social sharing, so that each chapter shared will have a unique description, auto-generated based on the content.

Specify your book's source repository on GitHub as the \texttt{repo} in the \texttt{\_output.yml} file, which allows users to view each chapter's source file or suggest an edit. Read more about the features of this output format here:

\url{https://pkgs.rstudio.com/bookdown/reference/bs4_book.html}

Or use:

\begin{Shaded}
\begin{Highlighting}[]
\NormalTok{?bookdown}\SpecialCharTok{::}\NormalTok{bs4\_book}
\end{Highlighting}
\end{Shaded}


  \bibliography{book.bib,packages.bib}

\end{document}
